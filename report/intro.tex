\newpage
\section*{Введение}
\addcontentsline{toc}{section}{Введение}

Локализация человеческого лица с последующей идентификацией давно находится в списке наиболее важных задач для исследователей в области систем машинного зрения и искусственного интеллекта. Множество исследований, проводимых ведущими научными центрами, так и не позволило создать универсальную систему компьютерного зрения, способную к локализации и распознаванию лица человека при различных условиях.

Причиной тому явилось сразу множество факторов, среди которых:
\begin{itemize}
\item высокая вариативностью лиц, обусловленной анатомическими особенностями людей;
\item различный уровень освещенности объектов, зависящих от типа, количества и характеристик направленности источников света;
\item необходимость обнаружения лиц, имеющих различное пространственное положение.
\end{itemize}

Зачастую особенности системы, использующей локализацию и распознавание лиц может накладывать дополнительные ограничения на скорость работы (близкой к реальному времени), аппаратным ресурсам (процессор, объём памяти) и экономичности (аккумулятора).

Кроме скоростных характеристик, от алгоритма требуется обеспечение малого (порядка 5\%) количества ложных распознаваний. В системах, реализующих существующие методы распознавания, при увеличении уровня распознаваний свыше 90\% наблюдается существенный рост числа ложных решений, что затрудняет их практическое использование. Прежде чем распознавать лицо, необходимо убедиться в его присутствии на изображении. Для чего применяются известные методы обнаружения и распознавания лиц на изображениях (метод главных компонент, нейронные сети, метод опорных векторов). Результативность применения метода определяется спецификой решаемой задачи. Поэтому построение метода распознавания лиц, обеспечивающего высокий уровень достоверности решения при отсутствии ограничений на исходные изображения, является весьма актуальной задачей.

Целью данной работы является обзор основных методов обнаружения лиц (под обнаружением лиц на изображении будем понимать процесс локализации областей изображения, содержащих лица людей; границы искомых областей в в общем случае размыты, однако чаще всего подразумевается минимальный описывающий прямоугольник), обеспечивающих повышение достоверности распознавания объектов анализа, снижение уровня ложных распознаваний, уменьшение времени обучения классификатора и времени предварительной обработки изображения.